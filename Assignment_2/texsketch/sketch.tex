\documentclass{article}

\usepackage{amsmath} % for math symbols and equations
\usepackage{amsfonts} % add this line to import the missing package

\newcommand{\bb}{b^{(2)}}
\newcommand{\ab}{a^{(2)}}
\newcommand{\zc}{z^{(3)}}
\newcommand{\Wb}{W^{(2)}}


\begin{document}

\section*{Q2}
\subsection*{(b)}
Case $N = 1$\\
Note that $\frac{\partial \zc}{\partial \Wb} \in \mathbb{R}^{K\times H\times K}$ is a tensor.
To simplify things we can look at the derivative of $\zc$ w.r.t a single element in $\Wb$: 
$\frac{\partial \zc}{\partial \Wb_{ij}}$.
We know that $\zc = \Wb\ab + \bb$ where $\ab \in \mathbb{R}^{H}$ and $\Wb \in \mathbb{R}^{K\times H}$.
It can also be written as
\begin{align*}
    \zc_i &= \left(\sum_{j=1}^{H} \Wb_{ij} \ab_j\right) + \bb_i
\end{align*}
apparently
\begin{align*}
    \frac{\partial \zc_i}{\partial \Wb_{kj}} =
    \begin{cases}
        \ab_j & \text{if } i = k\\
        0 & \text{otherwise}
    \end{cases}
\end{align*}
which is a vector filled with zero except for the $i$-th element.

By chain rule, we have
\begin{align*}
    \frac{\partial J}{\partial \Wb_{ij}} = \frac{\partial J}{\partial \zc} \cdot \frac{\zc}{\partial \Wb_{ij}}
\end{align*}
Note that $\frac{\partial J}{\partial \Wb_{ij}} \in \mathbb{R}$ since both $J$ and $\Wb_{ij}$ are scalars.
Also, most of the elements in $\frac{\zc}{\partial \Wb_{ij}}$ are zeros, except for the $i$-th element which is $\ab_j$.
Therefore, the result of the dot product is
\begin{align*}
    \frac{\partial J}{\partial \Wb_{ij}} = (\frac{\partial J}{\partial \zc})_i \ab_j
\end{align*}
We can express the Jacobian matrix $\frac{\partial J}{\partial \Wb}$ as
\begin{align*}
    \left(\frac{\partial J}{\partial \Wb}\right)_{ij} = \left(\frac{\partial J}{\partial \zc}\right)_i \ab_j
\end{align*}
Case $N > 1$\\
We have the upstream gradient $\frac{\partial J}{\partial \zc} \in \mathbb{R}^{N\times K}$ 
and $\ab \in \mathbb{R}^{N\times H}$ because the loss function is an average of $N$ losses.
Each element of $\frac{\partial J}{\partial \Wb}$ should have form 
\begin{align*}
    \left(\frac{\partial J}{\partial \Wb}\right)_{ij} = \frac{1}{N}\sum_{n}^{N}\left(\frac{\partial J}{\partial \zc}\right)_{ni} \ab_{nj}
\end{align*}
which can be expressed as a matrix multiplication:
\begin{align*}
    \frac{\partial J}{\partial \Wb} = \frac{1}{N}{\left(\frac{\partial J}{\partial \zc}\right)}^\top\ab
\end{align*}
verify the dimensions: ${\left(\frac{\partial J}{\partial \zc}\right)}^\top\in\mathbb{R}^{K\times N}$
so $\frac{\partial J}{\partial \Wb} \in \mathbb{R}^{K\times H}$ and $\Wb\in \mathbb{R}^{K\times H}$ 
which are the same.

Thus we have 
\begin{align*}
    \frac{\partial J}{\partial \Wb} = \frac{1}{N}{\left(\psi(z^{(3)}) - \Delta\right)}^\top\ab
\end{align*}
and with regularization
\begin{align*}
    \frac{\partial \tilde{J}}{\partial \Wb} = \frac{1}{N}{\left(\psi(z^{(3)}) - \Delta\right)}^\top\ab + 2\lambda \Wb
\end{align*}

\subsection*{(c)}
\newcommand{\zb}{z^{(2)}}
\newcommand{\Wa}{W^{(1)}}
\newcommand{\ba}{b^{(1)}}
In case of $N = 1$, from the forumla of $\zc_i$ above we can derive that
\begin{align*}
    \frac{\partial \zc_i}{\partial \bb_k} = \begin{cases}
        1 & \text{if } i = k\\
        0 & \text{otherwise}
    \end{cases} = I
\end{align*}
Thus the $N > 1$ case Jacobian matrix $\frac{\partial J}{\partial \bb}$ can be expressed as
\begin{align*}
    \frac{\partial J}{\partial \bb} &= \frac{\partial J}{\partial \zc} \cdot I\\
    &= \frac{\partial J}{\partial \zc}\\
    &= \frac{1}{N}\left(\psi(z^{(3)}) - \Delta\right)
\end{align*}
Similarly
\begin{align*}
    \frac{\partial \zc_i}{\partial \ab_j} = \Wb_{ij}
\end{align*}
so for $N > 1$ we have
\begin{align*}
\frac{\partial J}{\partial \ab_j} = \frac{\partial J}{\partial \zc}  \Wb = 
\frac{1}{N}\left(\psi(z^{(3)}) - \Delta\right) \Wb
\end{align*}
The ReLU function $\phi$ is element-wise i.e. $\phi : \mathbb{R}^H \mapsto \mathbb{R}^H$
thus for $N = 1$, $\frac{\partial \ab}{\partial \zb} \in \mathbb{R}^{H\times H}$ is a diagonal matrix with its diagonal elements
\begin{align*}
    \left(\frac{\partial \ab}{\partial \zb}\right)_{ii} = \begin{cases}
        1 & \text{if } \zb_{i} > 0\\
        0 & \text{otherwise}
    \end{cases} = \mathrm{diag}(\mathrm{sign(\zb)})
\end{align*}
It is a bit tricky for the $N > 1$ case
as the upstream gradient $\frac{\partial J}{\partial \ab}$ is a matrix of size $N\times H$
where each row $(\frac{\partial J}{\partial \ab})_n\in\mathbb{R}^{1\times H}$.
for each input/row we have 
\begin{align*}
    (\frac{\partial J}{\partial \zb})_n = (\frac{\partial J}{\partial \ab})_n \cdot \mathrm{diag}(\mathrm{sign(\zb_n)})
\end{align*}
with this upstream gradient $\frac{\partial J}{\partial \zb}$
the gradient for $\Wa$ and $\ba$ can be calculated in a similar way as $\Wb$ and $\bb$, literally
\begin{align*}
    \frac{\partial J}{\partial \Wa} = \frac{1}{N}{\left(\frac{\partial J}{\partial \zb}\right)}^\top X
\end{align*}
and 
\begin{align*}
    \frac{\partial J}{\partial \ba} = \frac{1}{N}\left(\frac{\partial J}{\partial \zb}\right)
\end{align*}
\end{document}
