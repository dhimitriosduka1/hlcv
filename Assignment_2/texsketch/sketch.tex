\documentclass{article}

\usepackage{amsmath} % for math symbols and equations
\usepackage{amsfonts} % add this line to import the missing package

\newcommand{\bb}{b^{(2)}}
\newcommand{\ab}{a^{(2)}}
\newcommand{\zc}{z^{(3)}}
\newcommand{\Wb}{W^{(2)}}

\begin{document}

\section*{Q2}
\subsection*{(b)}
Case $N = 1$\\
Note that $\frac{\partial \zc}{\partial \Wb} \in \mathbb{R}^{K\times H\times K}$ is a tensor.
To simplify things we can look at the derivative of $\zc$ w.r.t a single element in $\Wb$: 
$\frac{\partial \zc}{\partial \Wb_{ij}}$.
We know that $\zc = \Wb\ab + \bb$ where $\ab \in \mathbb{R}^{H}$ and $\Wb \in \mathbb{R}^{K\times H}$.
It can also be written as
\begin{align*}
    \zc_i &= \left(\sum_{j=1}^{H} \Wb_{ij} \ab_j\right) + \bb_i
\end{align*}
apparently
\begin{align*}
    \frac{\partial \zc_i}{\partial \Wb_{kj}} =
    \begin{cases}
        \ab_j & \text{if } i = k\\
        0 & \text{otherwise}
    \end{cases}
\end{align*}
which is a vector filled with zero except for the $i$-th element.

By chain rule, we have
\begin{align*}
    \frac{\partial J}{\partial \Wb_{ij}} = \frac{\partial J}{\partial \zc} \cdot \frac{\zc}{\partial \Wb_{ij}}
\end{align*}
Note that $\frac{\partial J}{\partial \Wb_{ij}} \in \mathbb{R}$ since both $J$ and $\Wb_{ij}$ are scalars.
Also, most of the elements in $\frac{\zc}{\partial \Wb_{ij}}$ are zeros, except for the $i$-th element which is $\ab_j$.
Therefore, the result of the dot product is
\begin{align*}
    \frac{\partial J}{\partial \Wb_{ij}} = (\frac{\partial J}{\partial \zc})_i \ab_j
\end{align*}
We can express the Jacobian matrix $\frac{\partial J}{\partial \Wb}$ as
\begin{align*}
    \left(\frac{\partial J}{\partial \Wb}\right)_{ij} = \left(\frac{\partial J}{\partial \zc}\right)_i \ab_j
\end{align*}
Case $N > 1$\\
We have the upstream gradient $\frac{\partial J}{\partial \zc} \in \mathbb{R}^{N\times K}$ 
and $\ab \in \mathbb{R}^{N\times H}$ because the loss function is an average of $N$ losses.
Each element of $\frac{\partial J}{\partial \Wb}$ should have form 
\begin{align*}
    \left(\frac{\partial J}{\partial \Wb}\right)_{ij} = \frac{1}{N}\sum_{n}^{N}\left(\frac{\partial J}{\partial \zc}\right)_{ni} \ab_{nj}
\end{align*}
which can be expressed as a matrix multiplication:
\begin{align*}
    \frac{\partial J}{\partial \Wb} = \frac{1}{N}{\left(\frac{\partial J}{\partial \zc}\right)}^\top\ab
\end{align*}
verify the dimensions: ${\left(\frac{\partial J}{\partial \zc}\right)}^\top\in\mathbb{R}^{K\times N}$
so $\frac{\partial J}{\partial \Wb} \in \mathbb{R}^{K\times H}$ and $\Wb\in \mathbb{R}^{K\times H}$ 
which are the same.

Thus we have 
\begin{align*}
    \frac{\partial J}{\partial \Wb} = \frac{1}{N}{\left(\psi(z^{(3)}) - \Delta\right)}^\top\ab
\end{align*}
and with regularization
\begin{align*}
    \frac{\partial \tilde{J}}{\partial \Wb} = \frac{1}{N}{\left(\psi(z^{(3)}) - \Delta\right)}^\top\ab + 2\lambda \Wb
\end{align*}

\end{document}
